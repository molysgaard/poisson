\documentclass{article}
\usepackage[utf8]{inputenc}
\usepackage{cancel} % for å kunne stryke ledd
\usepackage{amsmath} % for align* og matte generelt
\usepackage{amssymb}
\usepackage{graphicx}
\usepackage{float}
\usepackage[strict]{changepage}
\usepackage{bm}% for bold matte
\usepackage{subcaption}
\usepackage{xfrac} % for å kunne skrive 2/3 brøker
\usepackage{hyperref} % for URLS


\newcommand{\D}[2]{\ensuremath{\frac{\partial #1}{\partial #2}}}
\newcommand{\dd}[2]{\ensuremath{\frac{\partial^2 #1}{\partial #2^2}}}
\newcommand{\ddd}[2]{\ensuremath{\frac{\partial^3 #1}{\partial #2^3}}}

\title{Semester project \\ TMA4212 Numerical solution of differential equations by difference methods }
\author{Candidate no. 10059,10044 and 10012.}
\date{April 2013}

\begin{document}

\maketitle

\section{introduction}
The first use of comptuers was for doing simple calculations at a higher rate
than a human possibly could. Today computers are no longer just calculators but
present everywhere around us, intergrating to our lives to a bigger and bigger extent.
Even though we have past the era of big calculators scientific computing and the use
of computers for doing ever bigger calculations is just increasing. Today it is
often much cheaper to do numerical experiments, simulation the nature, instead of
setting up a real experiment. Numerical experiments are usually cheaper, less
dangorous and easier to analyse the results from. Fields that use this extencively
are military, for nuclear explosion simulations and metrologists for simulation the weather.

Almost all computer aided design is today tested numerically before any physical model is made.
Planes are tested for their flight caractheristics, cars for their aeorodynamic properties, enignes
for their thermodynamics etc. All of these applications are very computational intensive and on the
really big scale, like weather forcasts, they require immense computational power, clever algorithms
and programming techniques. This report covers a sample problem in scientific computing. Solving the
Poisson equation for a big problem size. This requires an approach very different from a naive
sequential implementation to get good performance.

\section{Problem description}
\label{sec:sol_steps}
Poisson's equation is given as
\[
-\Delta u = f
\]
where $\Delta = \nabla^2$ is the Laplace operator, u and f are real or complex valued functions in a
euclidian space. The equation is often written as
\[
-\nabla^2 u = f.
\]
In two dimensions the equation can be written as
\[
-\left( \frac{\partial^2}{\partial x^2} + \frac{\partial^2}{\partial y^2} \right) u(x,y) = f(x,y).
\]

In section \ref{sec:meth_der} we derive the chosen solution method of this equation. The mathematics are
lengthy and it suffices for the parallel analyzis to know that one ends up with a three step solution
process.

\paragraph{step 1}
Form the matrix matrix product
\[
\tilde{G} = Q^\top G Q.
\]
\paragraph{step 2}
solve
\[
\Lambda \tilde{U} + \tilde{U} \Lambda = \tilde{G}
\]
or
\[
\tilde{u}_{i,j} = \frac{\tilde{g}_{i,j}}{\lambda_i + \lambda_j}.
\]
\paragraph{step 3}
Compute the matrix matrix product
\[
U = Q \tilde{U} Q^\top.
\]

Where $G$ and $U$ is a discrete version of $h^2 f$ and $u$ respectively, $Q$ is the eigen vectors of the Laplace operator,
$\lambda$ is a diagonal matrix of the eigen values of the discretized Laplace operator and $\tilde{A} = Q^\top A Q$.

\section{Computational complexity}
The asymptotic complexity of this method is bounded by the most expensive step
in the algorithm.
Given a 2D grid of size $n-1 \times n-1$ we can denote the following complexity to each step.
\paragraph{step 1}
2 matrix matrix products each have complexity $O(n^3)$.
\paragraph{step 2}
$n^2$ constant opperations, $O(n^2)$.
\paragraph{step 3}
2 matrix matrix products each have complexity $O(n^3)$.

This means that this solution method has $O(n^3)$ complexity and a serial implementation
will have asymptotic running time of $n^3$ of the problem size $n$. Here the problem size
\[
n \leq \max(n_x,n_y)
\]
where $n_x$ and $n_y$ is the number of grid points in our finite difference approximation
in the x and y direction respectively.

\section{Survey of alternate solution method}
The simplest method of solving the Poisson equation is to apply a 5 point stencil on your domain
iteratively. This means that each cell in the grid gives a fraction of its own value to its neighbors.
Then this is done for each cell iteratively untill the grid does not change more than a given threshold.

This method has the same asymptotic operation cost, $O(n^3)$ and better memory requirement $O(n^2)$.
However it does not lend itself well to paralellization. To paralellize this one would split
up the domain in blocks that each node can take care of. The five point stencil would then be
run in paralell on each of the nodes. At the end of each iteration phase a global
syncronization would have to occur for all the nodes in the compute grid to know the boundary values
to its own domain. This would lead to $O(n)$ all to all communcation steps each of a cost of
\[
4\frac{n}{\sqrt{p}}
\]
elements, where $p$ is the number of compute nodes the domain is distributed on.
This syncronization would make the algorithm way to slow for any practical size
of the problem compared to the direct diagonalization method.

\section{Analysis of paralellizable sections.}
The presented solution method (section \ref{sec:sol_steps}) is not a obvious candidate for paralellization
since a large majority of the time is used doing matrix multiplications.
This means that to make this algorithm scale well a large scale paralellized
matrix multiply has to implemented. A naive paralellization on a single SMP machine
was rejected as it would not let the algoritm scale to the huge scales that are realistic
in applications such as weather forcasts.
After some research it was found that that there exists a algorithm for this, \cite{summa}, discovered in 1997.

\subsection{The SUMMA algorithm}
TODO: Explain the SUMMA algorithm

\section{Wall time analyzis}
\subsection{Serial implementation runtime}
plot the run time for a serial implementation

\subsection{Parallel implementation run time}
plot the run time for a paralell implementation

\subsection{Parallel CUDA implementation run time}
plot the run time for a CUDA paralell implementation

\subsection{Speedup}
plot speedup


\section{Journal}
\begin{enumerate}
	\item Started with Haskell. Managed to create a sequential that was on par with C and had much better readability and consiseness.
	\item Started looking at the accelerate library for Haskell. Managed to implement a matrix multiply in it but it lacked performance because of the non-flat nature of the matrix multiply algorithm in accelerate. This made it solver than a sequential C implementation.
	\item Contacted author of accelerate to see if there was any way to mend the problems but it turned out to be a limitation in the embedded language that accelerate is. There is no way to express the type of computation that matrix multiply is effeciently. This is because of the intermidiat dot-product you calculate when doing matrix multiply. Accelerate is extremely good at things that it's made for, like stencil computations. This was also looked at but because the application was to be solving extremely large linear systems they would not fit in the memory of one GPU.
	\item First version of the SUMMA algorithm used FORTAN routines for the matrix multipyl and array copying. Since FORTRAN ararys are column-major instead of row-major as in C this lead to unpredictable and really hard to debug problems in the algorithm.
	\item Scattering and gathering the matrices from the master node proved prone to error. Not because of bad libraries or programming errors. The challenge was creating the correct mapping in a general case from a contigous array to smaller blocked contigous arrays at the low abstraction level that C works.
	\item Implementing a really fast sequential version is not trivial since most of the tools for fast numerical code, BLAS, LAPACK etc. do paralellization without you knowing it. For example the first sequential version of the program used the cblas\_dgemm routine. When this routine was profiled it was experienced that the routine forked off several threads to do it's work, thus these library functions could not be used for a sequential implementation.
\end{enumerate}



\begin{thebibliography}{9}

\bibitem{summa}
  Robert A. van de Geijn and Jerrell Watts (1997)
  \emph{SUMMA: Scalable Universal Matrix Multiplication Algorithm}
  \url{http://www.netlib.org/lapack/lawnspdf/lawn96.pdf}

\bibitem{blas}
  \url{http://www.netlib.org/blas/}

\bibitem{lapack}
  \url{http://www.netlib.org/lapack/}

\end{thebibliography}

\end{document}

